\chapter{Исследовательская часть}

В данном разделе представлены примеры работы программного обеспечения, а также описан эксперимент по проведению нагрузочного тестирования разработанной системы.


\section{Пример работы программного обеспечения}

На рисунках 4.1-4.4 приведены примеры использования разработанного приложения.

\img{80mm}{6}{Пример запроса на чтение}
\img{140mm}{9}{Пример запроса на чтение}
\img{50mm}{7}{Пример запросов на неаутентифицированное изменение данных}
\img{12mm}{8}{Пример запроса на аутентифицированное изменение данных}



\section{Постановка эксперимента}

Целью эксперимента является проведение нагрузочного тестирования разработанной системы. Критерием оценки выступили результаты аналогичного тестирования схожей системы, но в которой не реализован ни один из методов масштабирования БД.


Технические характеристики ЭВМ, на котором выполнялись исследования:
\begin{itemize}
	\item процесор Intel Core i5-8250U 1.6ГГц (8 ядер);
	\item 8 Гб оперативной памяти;
	\item OC Linux Mint 19.3.
\end{itemize}
В ходе проведения эксперимента ЭВМ была подключена к сети электропитания.


Эксперимент был поставлен по следующему сценарию:
\begin{itemize}
	\item был создано два типа подключений: гость (guest), который имел большой набор разнообразных запросов на чтение, причем некоторые запросы отправлялись чаще других; и администратор (admin), который создает и изменяет записи в базе данных. 90\% всех запросов составляют запросы гостя и, соответственно, 10\% - администратора;
 	\item каждую секунду количество подключений увеличивается на 10, пиковая нагрузка - 1000 пользователей.
\end{itemize}


В таблице 4.1 приведены агрегированные результаты системы с и без применения методов DIA.
\begin{table}[h!]
	\caption{Агрегированные результаты тестирования}
	\resizebox{\columnwidth}{!}{
	\begin{tabular}{|c |p{2cm} p{2cm} p{2cm} p{2cm} p{2cm} |}
		\hline
		& Всего запросов отправлено & Число ошибок &  Cреднее время отклика (мс) & RPS & Failures/s\\ 
		\hline
		\textbf{Базовая система} & 2728	& 1902 & 14055& 23.3 &	16.3 \\
		\hline
		\textbf{Система DIA} & 5670 & 463 & 5713 & 49.9 & 4.1 \\
		\hline
	\end{tabular}}
\end{table}

В таблице 4.2 приведены результаты измерения времени отклика системы.
\begin{table}.
	\caption{Время отклика системы (мс)}
	\resizebox{\columnwidth}{!}{
		\begin{tabular}{|c |p{2cm} p{2cm} p{2cm} p{2cm} p{2cm} |}
			\hline
			& 50\% процентиль & 80\% процентиль & 90\% процентиль & 95\% процентиль & 100\% процентиль \\ 
			\hline
			\textbf{Базовая система} & 8300 & 16000	& 50000 & 60000 & 75000 \\
			\hline
			\textbf{Система DIA} & 3000 & 5800 & 10000 & 21000 & 71000 \\
			\hline
	\end{tabular}}
\end{table}

На рисунках 4.5 и 4.6 приведены графики результатов проведенного тестирования базовой системы, и системы с применением методов DIA, соответственно.

\img{195mm}{10}{Графики, построенные по результатам тестирования базовой системы}
\img{195mm}{11}{Графики, построенные по результатам тестирования DIA-системы}


Относительно большое число ошибок связано со вычислительной мощностью машины, на которой проводилось тестирвоание, поскольку на одной ЭВМ были запущены одновременно Python-сервер, сервер Redis, три реплики БД, а так же тестирующее ПО.

Применение методов DIA позволило увлечить RPS системы более чем в два раза, уменьшить в 2.5 раза среднее время отклика и в 4 раза число ошибок в секунду - что свидетельствует о том, что разработанная система эффективнее справляется с относительно большой нагрузкой. 

Как следует из полученных графиков, число ошибок в секунду в базовой системе было примерно на уровне RPS, что означает, почти каждый новый запрос в систему возвращал ошибку. В то время как число ошибок DIA-сервиса держалось на одном уровне по мере увеличения нагрузки.

%Отметим, что в DIA-системе успешно был обработан каждый запрос на изменение данных, что является критичным для некоторых видов сервисов.




\section*{Вывод}

В результате сравнения разработанной с применением методов и механизмов масштабирования БД системы с базовой, была доказана эффективность примененных решений для создания DIA.