\chapter{Технологическая часть}

В данном разделе представлены средства разработки программного обеспечения, а также детали реализации.

\section{Выбор технических средств}

Для взаимодействия серверного программного обеспечения с СУБД былы выбрана технология object–relational mapping (ORM). Она обеспечивает работу с данными в терминах классов, а не таблиц данных, что ускоряет процесс разработки. Использование ORM позволяет в будущем легко подменять СУБД, не перписывая код, что является весомым аргументом в пользу использования этой технологии. Помимо всего выше перечисленного, ORMs используют параметризованные SQL-выражения, который защищают от прямых атак SQL-инъекциями.

В качестве языка разработки выбран язык программирования Python 3 \cite{python}. Данный выбор обусловлен простотой использования, скоростью написания кода и большим количеством всевозможных библиотек, необходимых для разработки ПО. Редактор кода - VS Code \cite{vscode}. Выбор среды обусловлен большим количеством доступных плагинов платформы, которые существенно облегчают и ускоряют процесс написание кода.

В качестве Web-framework был выбран Flask \cite{flask}, поскольку он предлагает широкий функционал для написания веб-приложений на языке Python 3.

Для проведения нагрузочного тестирования была выбрана библиотека Locust \cite{locust}. Этот выбор обусловлен тем, что данная библиотека предоставляет удобный и исчерпывающий отчет проведенного тестирования, а так же визуализирует полученные результаты.



\section{Детали реализации}

\subsection{Взаимодействие с разработанным приложением}

В таблице 3.1 приведен список реализованных HTTP-запросов для взаимодействия с базой данных.

\begin{table}[h!]
	\caption{Разработанные запросы к базе данных}
	\resizebox{\columnwidth}{!}{
		\begin{tabular}{| c | p{11cm} | c |}
			\hline
			\textbf{HTTP-метод} & \textbf{Действие} & \textbf{URL-путь} \\ 
			\hline
			GET & Получить информацию о самолете(-ах) & /api/aircrafts \\
			\hlinefill
			POST & Добавить запись о самолете & /api/aircrafts \\
			\hlinefill
			PUT & Изменить запись о самолете & /api/aircrafts \\
			\hlinefill
			DELETE & Удалить запись о самолете & /api/aircrafts \\
			\hline
			GET & Получить информацию об аэропорте(-ах) & /api/airports \\
			\hlinefill
			POST & Добавить запись об аэропорте & /api/airports \\
			\hlinefill
			PUT & Изменить запись об аэропорте & /api/airports \\
			\hlinefill
			DELETE & Удалить запись об аэропорте & /api/airports \\
			\hline
			GET & Получить информацию об авиакомпании(-ях) & /api/airlines \\
			\hlinefill
			POST & Добавить запись об авиакомпании & /api/airlines \\
			\hlinefill
			PUT & Изменить запись об авиакомпании & /api/airlines \\
			\hlinefill
			DELETE & Удалить запись об авиакомпании & /api/airlines \\
			\hline
			GET & Получить информацию о перелетах(-ах) & /api/flights \\
			\hlinefill
			POST & Добавить запись о перелете & /api/flights \\
			\hlinefill
			PUT & Изменить запись о перелете & /api/flights \\
			\hlinefill
			DELETE & Удалить запись о перелете & /api/flights \\
			\hline
			GET & Получить информацию об авиакомпании-операторе воздушного судна & /api/ac-operator \\
			\hline
			GET & Получить среднее время задержки рейсов заданной авиакомпании & /api/avg-time \\
			\hline
	\end{tabular}}
\end{table}

Для GET-запросов предусмотрены следующие аргументы:
\begin{itemize}
	\item /api/aircrafts - для выборки самолетов, удовлетворяющих следующим параметрам:
	\begin{itemize}
		\item tail-no - номер регистрации воздушного судна;
		\item mfr - компания-производитель;
		\item model - модель самолета;
		\item younger (older) -  поступил в эксплуатацию после (до) указанной даты (в формате ГГГГ-ММ-ДД);
		\item limit - число возвращенных записей (по умолчанию 10);
	\end{itemize}
	\item /api/airports - для выборки аэропортов, удовлетворяющих следующим параметрам:
	\begin{itemize}
		\item iata - трехсимвольный код аэропорта в международной кодификации;
		\item city, state, country - город, штат (область) и страна, соответственно, в которой находится аэропорт;
		\item limit - число возвращенных записей (по умолчанию 10);
	\end{itemize}
	\item /api/airlines - для выборки авиакомпаний, удовлетворяющих следующим параметрам:
	\begin{itemize}
		\item id - двухсимвольный код авиакомпании в международной кодификации;
		\item limit - число возвращенных записей (по умолчанию 10);
	\end{itemize}
	\item /api/flights - для выборки перелетов, удовлетворяющих следующим параметрам:
	\begin{itemize}
		\item id - уникальный идентификатор записи;
		\item flightid - номер рейса;
		\item date - дата совершения перелета (в формате ГГГГ-ММ-ДД);
		\item dow - день недели (воскресенье - 0, суббота - 6);
		\item tail-no - номер регистрации воздушного судна, выполнявшего рейс;
		\item from, to - код аэропорта вылета и прилета, соответственно;
		\item limit - число возвращенных записей (по умолчанию 10);
	\end{itemize}
	\item /api/ac-operator - для получения информации об операторе заданного самолета:
	\begin{itemize}
		\item  tail-no - номер регистрации воздушного судна;
	\end{itemize}
	\item /api/avg-time - для получения информации о среднем времени задержок рейсов заданной авиакомпании:
	\begin{itemize}
		\item  id - двухсимвольный код авиакомпании в международной кодификации.
	\end{itemize}
\end{itemize}

Запросы на сервер можно посылать как из браузера (URL), так и с помощью специальных утилит терминала (например, curl). Ответ от сервера приходит в виде JSON-объекта. Поле <<ok>> отражает, возникла ли ошибка при обработке запроса, а в поле <<result>> - непосредственно содержимое ответа.


\subsection{Листинги скриптов и подпрограмм}

Для реализации Master-Slave репликации на каждый из контейнеров была установлена ОС Ubuntu \cite{ubuntu} и актуальная (такая же как и на Master-реплике) версия PostgreSQL \cite{postgres}. Адреса хостов в локальной сети были получены с помощью утилиты ifconfig. В приложении на листингах 4.1 и 4.2 приведена настройка Master и Slave-реплик базы данных, путем редактирования системных файлов-конфигураторов СУБД (postgresql.conf и pg\_hba.conf, в частности). 

На листингах 4.3-4.5 в приложении приведены реализации механизмов масштабирования БД, рассмотренных в конструкторском разделе:
\begin{itemize}
	\item на листинге 4.3 - балансировка запросов между репликами БД;
	\item на листинге 4.4 - SQL-скрипт создания индексов в БД;
	\item на листинге 4.5 - реализация алгоритма 2Q с помощью средств СУБД Redis.
\end{itemize}

Для GET-запроса /api/ac-operator была реализована функция на языке plpgsql. Реализации функции на языке plpgsql приведена в приложении на листинге 4.6.

В приложении на листинге 4.7 приведена реализации рассмотренной в конструкторском разделе ролевой модели на уровне БД. Роль postgres - созданная СУБД роль по умолчанию никак не была изменена.

\section*{Вывод}

В данном разделе были подробно рассмотрены технические средства для написания программного обеспечения, а также детали реализации ПО (интерфейс взаимодействия, листинги скриптов и подпрограмм).