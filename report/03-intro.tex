\chapter*{Введение}
\addcontentsline{toc}{chapter}{Введение}




За последнее десятилетие объем интернет-трафика колоссально вырос: коммерческие компании обрабатывают все больше данных и трафика, что, в свою, очередь, вынуждает их создавать новые инструменты, подходящие для эффективной работы в подобных масштабах.

Высоконагруженные данными приложения (data-intensive applications, DIA) открывают новые горизонты возможностей благодаря использованию современных технологических усовершенствований. Говорят, что приложение является высоконагруженным данными (data-intensive), если те представляют основную проблему, с которой оно сталкивается, — качество данных, степень их сложности или скорость изменений, — в отличие от высоконагруженного вычислениями (compute-intensive), где узким местом являются циклы CPU \cite{oreilly}. Далее высоконагруженное данными приложение будем называть просто высоконагруженным.

Инструменты и технологии, обеспечивающие хранение и обработку данных с помощью DIA, играют важную роль в разработке современных программных продуктов, а знание и умение применят те или иные технологии DIA в некоторой степени является показателем профессионализма разработчика.

Целью данного проекта является моделирование высоконагруженной системы c применением механизмов и алгоритмов масштабирования базы данных на примере RESTful \cite{rest-api} приложения, предоставляющего данные об перелетах, а также эмпирическая оценка эффективности разработанной системы.
Для достижения поставленной цели в ходе работы требуется решить следующие задачи:

\begin{enumerate}
	\item провести анализ существующих алгоритмов и методов построения высоконагруженной системы, выбрать из них подходящие для выполнения проекта;
	\item с помощью выбранных методов разработать систему;
	\item выбрать технологии (СУБД, язык программирования, фреймворк) для реализации поставленной задачи;
	\item произвести нагрузочное тестирование системы.
\end{enumerate}