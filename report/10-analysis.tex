\chapter{Аналитическая часть}

В данном разделе проанализирована поставленная задача, а также рассмотрены различные способы ее реализации. 

\section{Постановка задачи}

В рамках курсового проекта необходимо реализовать высоконагруженное RESTful \cite{rest-api} приложение для поиска информации об авиаперелетах, и включить в него следующий функционал:
\begin{itemize}
 	\item предоставить возможность пользователем считывать данные с помощью GET-запросов на сервер;
 	\item разрешить вносить изменения в данные только администратору сервиса.
\end{itemize}

Предполагается, что подавляющее большинство запросов - запросы на чтение.


\section{Формализация данных}

База данных должна хранить следующую информацию:
\begin{enumerate}
	\item данные о самолетах, аэропортах, авиакомпаниях;
	\item информацию о рейсах, связанную с данными об аэропортах, самолетах и авиакомпаниях.
\end{enumerate}

В таблице 1.1 приведены выделенные категории данных и краткие сведения о них:
\begin{table}.
	\caption{Категории данных}
	\resizebox{\columnwidth}{!}{\begin{tabular}{|c | p{13cm} |}
		\hline
		\textbf{Категория} & \textbf{Сведения о категории} \\
		\hline
		Самолет & ID самолета (номер регистрации); модель, месяц и год ввода в эксплуатацию; фотография судна \\
		\hline
		Аэропорт & ID аэропрта (в международной кодировке); расположение: город, провинция (штат), страна, координаты; \\
		\hline
		Авиакомпания & ID авиакомпании (в международной кодировке); название; немного контактной информации (номер телефона, адрес центрального офиса и т.д.) \\
		\hline
		Рейс & ID рейса; коды аэропортов вылета и прилета; ID самолета, выполнившего рейс; номер рейса (ID авиакомпании и числовой код); время в пути; преодолимое расстояние; статус рейса (задержан, отменен и т.д.). \\
		\hline
	\end{tabular}}
\end{table}


\section{Типы пользователей}

Для изменения данных в базе необходимо иметь соответствующие права администратора - необходима аутентификация суперпользователя. Таким образом в системе различается два вида пользователей: неаутентифицированные (<<гости>>) и аутентифицорованные (администраторы). В таблице 1.2 приведен функционалл каждого типа пользователя.

\begin{table}[h!]
	\caption{Функционал пользователя}
	\resizebox{\columnwidth}{!}{\begin{tabular}{|c | p{7cm} |}
			\hline
			\textbf{Тип пользователя} & \textbf{Функционал пользователя} \\
			\hline
			Неуаутентифицированный пользователь & Чтение данных \\
			\hline
			Аутентифицированный пользователь & Чтение, добавление, удаление, изменение данных  \\
			\hline
	\end{tabular}}
\end{table}


\section{Обзор существующих СУБД}

Система управления базами данных, сокр. СУБД — совокупность программных и лингвистических средств общего или специального назначения, обеспечивающих управление созданием и использованием баз данных \cite{iso-db}.

Основными функциями СУБД являются:
\begin{itemize}
	\item управление данными во внешней памяти, в оперативной памяти с использованием дискового кэша;
	\item журнализация изменений, резервное копирование и восстановление базы данных после сбоев;
	\item поддержка языков БД.
\end{itemize}



\subsection{Классификация СУБД по способу хранения}

По способу хранения выделяют два вида баз данных:
\begin{enumerate}
	\item СУБД с построчным хранением данных;
	\item СУБД с колоночным хранением данных.
\end{enumerate}

В СУБД с построчным хранением данных записи хранятся в памяти построчно. Для таких систем характерно большое количество коротких транзакций с операциями вставки, обновления и удаления данных. Зачастую их используют в транзакционных системах (OLTP-системы). Основными задачами транзакционных систем являются:
\begin{itemize}
	\item быстрая обработка запросов на добавление, изменение, удаление записей;
	\item поддержание целостности данных.
\end{itemize}
Показателем эффективности системы является количество транзакций в секунду.


Записи в СУБД колоночного типа представляются в памяти по столбцам.
Данный тип хранения данных нашел применяется в аналитических системах, для которых
характерно относительно небольшое количество транзакций, а запросы на чтение зачастую вычислительно сложны и включают в себя агрегацию данных. Время отклика является показателем эффективности аналитических систем.


\subsection{Классификация СУБД по модели данных}


Модель данных — это абстрактное, самодостаточное, логическое определение объектов, операторов и прочих элементов, в совокупности составляющих абстрактную машину доступа к данным, с которой взаимодействует пользователь. Эти объекты позволяют моделировать структуру данных, а операторы — поведение данных \cite{dbinfo}.

Выделяют 2 основных типа моделей организации данных:
\begin{itemize}
	\item реляционная (SQL);
	\item нереляционная (NoSQL).
\end{itemize}

Реляционная модель данных является совокупностью данных и состоит из набора двумерных таблиц. При табличной организации отсутствует иерархия элементов. Таблицы состоят из строк – записей и столбцов – полей. На пересечении строк и столбцов находятся конкретные значения. Для каждого поля определяется множество его значений. За счет возможности просмотра строк и столбцов в любом порядке достигается гибкость выбора подмножества элементов.
Таблицы реляционной СУБД могут быть связаны между собой с помощью внешних ключей (англ. foreign key), таким образом образуя некоторые отношения в базе данных.

Реляционные базы данных обеспечивают набор свойств ACID: атомарность, непротиворечивость, изолированность, надежность. Так, атомарность требует, чтобы транзакция выполнялась полностью или не выполнялась вообще; непротиворечивость означает, что сразу по завершении транзакции данные должны соответствовать схеме базы данных; изолированность требует, чтобы параллельные транзакции выполнялись отдельно друг от друга, а надежность подразумевает способность восстанавливаться до последнего сохраненного состояния после непредвиденного сбоя в системе или перебоя в подаче питания.

Реляционная модель является удобной и наиболее широко используемой формой представления данных. Примером популярных реляционных СУБД являются PostgreSQL \cite{postgres}, Oracle \cite{oracle}, Microsoft SQL Server \cite{mssql}.

На рисунке 1.1 приведена схема реляционной модели данных.
\img{30mm}{1}{Схема реляционной модели данных}


В отличие от реляционных СУБД, нереляционные базы данных не имеют одного общего формата. Данные в NoSQL СУБД могут хранится в виде пар ключ-значение (Redis \cite{redis}, LevelDB \cite{leveldb}), документов (MongoDB, Tarantool), графа (Neo4j, Giraph).

Как правило, базы данных NoSQL предлагают гибкие схемы, что позволяет осуществлять разработку быстрее и обеспечивает возможность поэтапной реализации. Благодаря использованию гибких моделей данных БД NoSQL хорошо подходят для частично структурированных и неструктурированных данных.

NoSQL базы данных зачастую предлагают компромисс, смягчая жесткие требования свойств ACID ради более гибкой модели данных, которая допускает горизонтальное масштабирование.




\section{Обзор существующих методов и технологий DIA}

\subsection{Индексирование}

Чтобы выполнить SELECT-запрос с предикатом WHERE система должна последовательно ряд за рядом (строка за строкой) просканировать таблицу целиком, чтобы найти все, удовлетворяющие условию вхождения. Это крайне неэффективная стратегия поиска, если в таблице много строк, и гораздо меньше (например, всего пара) целевых значений, которые надо вернуть. Но если создать в системе индекс на атрибуте, по которому проверяется условие в SELECT-запросе, СУБД сможет использовать более эффективные стратегии поиска. Так, например, системе может потребоваться всего несколько шагов в глубину по дереву поиска.

Схожий подход используется в большинстве документальной литературе и энциклопедиях: основные термины и концепты, которые часто ищутся читателями, вынесены в алфавитный указатель в конце книги. Читатель может как прочитать всю книгу целиком и найти интересующую его информацию (последовательный поиск), так и открыть алфавитный указатель и быстро пролистать страницы до нужной (поиск по индексу).

Индексы могут также положительно сказываться на производительности UPDATE и DELETE запросов, могут использоваться в JOIN-поисках \cite{indexes}. 
 



\subsection{Кэширование}

Зачастую приходится выполнять сложные SELECT-запросы: c JOIN-ом больших таблиц, оконными функциями и т.д. Если такой запрос СУБД выполняет относительно часто, тогда имеет смысл сохранять (на некоторое время) результат выполнения такого запроса, чтобы при следующем обращении в систему с таким же запросом, система вернула сохраненный результат, а не выполняла вычислительно сложный запрос еще раз. Это называется кэшированием результата запроса.

В качестве кэша часто используются key-value хранилища, такие как Redis \cite{redis} и LevelDB \cite{leveldb}.

Существует множество стратегий (алгоритмов) кэширования. Самым популярным и простым алгоритмом кэширования является алгоритм LRU (от англ. Least Recently Used) \cite{lru}. Суть этого алгоритма заключается в том, что из кэша вытесняются значения, которые дольше всего не запрашивались. Реализуется простой очередью (FIFO). Новые и запрошенные объекты перемещаются в начало очереди, соответственно, в конце очереди оказываются самые непопулярные объекты.

Несмотря на свою простоту и легкость создания, кэш LRU имеет критический недостаток: при добавлении новых объектов в переполненную очередь-LRU  из очереди может быть вытеснен <<теплый>> (часто запрашиваемый) объект. Чтобы решить эту проблему, был предложен алгоритм 2Q \cite{2q} (2 queue - англ. 2 очереди). Исходя из названия алгоритма, он использует вместо одной очереди две: в первой (<<горячей>>) очереди реализован обычный алгоритм LRU, а во второй (<<теплой>>) - простая очередь. Новые объекты помещаются в <<теплую>> очередь и постепенно сдвигаются к концу очереди, после чего вытесняются. Однако если пока объект находился <<теплой>> очереди, к нему произошло обращение, то запрашиваемый объект перемещается в <<горячую>> очередь.



\subsection{Репликация}

Репликация - это создание копий БД на разных серверах, но с разными привилегиями (правами). Создав несколько копий БД, можно балансировать нагрузку между хостами.

Выделяют два типа реплик:
\begin{itemize}
	\item Master-реплики - БД, в которой можно выполнять любые операции над данными.
	\item Slave-реплики - БД, из которой только можно считать информацию.
\end{itemize}

Репликация Master-Slave - это репликация, при которой единовременно существует один мастер и множество слейвов. Запросы на чтение равномерно распределяются между слейвами, в то время как запросы на модификацию данных поступают на мастер. Обновленная информация в мастере распространяется на слейвы, чтобы поддерживать их актуальность.Если выходит из строя мастер, нужно переключить все операции (и чтения, и записи) на слейв. Таким образом он станет новым мастером. А после восстановления старого мастера, настроить на нем реплику, и он станет новым слейвом.

Преобладание запросов на чтение, над запросами на изменение является ключевым критерием эффективности применения стратегии Master-Slave репликации \cite{replication}. Также некоторые из слейв-БД могут быть использованы в качестве резервной копии системы. 

Master-Master репликация подразумевает наличие нескольких мастер-узлов. Это накладывает свои ограничения на систему: выход из строя одного из серверов практически всегда приводит к потере каких-то данных. Последующее восстановление также сильно затрудняется необходимостью ручного анализа данных, которые успели либо не успели скопироваться. 



\subsection{Шардирование}

В жизненном цикле программного продукта может возникнуть ситуация, когда вертикальной стратегии масштабирования (путем наращивания вычислительной мощности хостов кластера: дисков, памяти и процессоров) на репликах может оказаться недостаточно. В этом случае можно разделить огромные таблицы данных на части и распределить их по репликам. В сущности, в этом и заключается смысл шардирования данных.

Выделяют три стратегии шардирования:
\begin{enumerate}
	\item вертикальное шардирование - поколоночное деление таблиц;
	\item горизонтальное шардирование - построчное деление таблиц ;
	\item диагональное шардирование - как следует из названия, это комбинация вертикального и горизонтального подхода.
\end{enumerate}

Шардирование обеспечивает несколько преимуществ, главное из которых — снижение издержек на обеспечение согласованного чтения, которое для ряда низкоуровневых операций требует монополизации ресурсов сервера баз данных, внося ограничения на количество одновременно обрабатываемых пользовательских запросов, вне зависимости от вычислительной мощности используемого оборудования). В случае шардинга логически независимые серверы баз данных не требуют взаимной монопольной блокировки для обеспечения согласованного чтения, тем самым снимая ограничения на количество одновременно обрабатываемых пользовательских запросов в кластере в целом \cite{sharding}. 


\section*{Вывод}

Изучив существующие виды СУБД, было принято решение использовать реляционную СУБД для хранения данных о перелетах, так как имеющиеся данные организованы в таблицы, а также имеют связь между собой, которую удобно реализовать с помощью внешних ключей; нереляционную СУБД, в которой данные хранятся в виде пар ключ-значение, в качестве кэша  запросов, поскольку такая структура хранения удобна для кэширования результатов исполнения запросов к БД: ключом является, например, SQL-выражение запроса, а значением - закэшированные данные.

Проведенный анализ существующих методов и технологий DIA позволяет сделать вывод, что для решения поставленной задачи наилучшим решением является применение каждого из подходов, за исключением шардирования, поскольку объем данных, используемый в проекте относительно невелик:
\begin{itemize}
	\item создание индексов - для ускорения выполнения SELECT-запросов;
	\item кэширование - для сохранения частых запросов; в качестве кэша принято решение использовать алгоритм 2Q, поскольку в нем решена основная проблема алгоритма LRU - вытеснение <<теплых>> данных;
	\item репликация Master-Slave, поскольку по условию задания SELECT-запросы преобладают в количестве над запросами на изменение данных.
\end{itemize}

Определившись с методами и алгоритмами, возникает следующая проблема: адаптация всего этого под конкретную задачу. Подробно это, а также детали реализации в следующем разделе.


















